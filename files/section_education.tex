% YAAC Another Awesome CV LaTeX Template
%
% This template has been downloaded from:
% https://github.com/darwiin/yaac-another-awesome-cv
%
% Author:
% Christophe Roger
%
% Template license:
% CC BY-SA 4.0 (https://creativecommons.org/licenses/by-sa/4.0/)
%Section: Work Experience at the top
\sectionTitle{Education}{\faGraduationCap}
%\renewcommand{\labelitemi}{$\bullet$}
\begin{experiences}
  \experience
    {now}{PhD in Astrophysics}{University of Padua}
    {Oct 2022}
    {Thesis: \textit{Binary compact object populations} \\
    Supervisor: Prof. Michela Mapelli  }{
    \begin{itemize}
        \item  With the stellar evolution software MESA, I study the correlation between stellar structure and mass transfer efficiency. I aim to extract fitting-formulae and tables that can be implemented by population-synthesis codes, allowing for more realistic simulations.
    \end{itemize}
   }
  \emptySeparator
  \experience
    {Sep 2022}{Master in Astrophysics and Cosmology}{University of Padua}
    {Oct 2020} 
    {Thesis: \textit{Wolf-Rayet -- black hole binaries as progenitors of binary black holes} \\
    Supervisor: Prof. Michela Mapelli; Co-Supervisor: Dr. Giuliano Iorio \\
    Grade: 110/110 cum laude}{
    \begin{itemize}
        \item I used the SEVN population-synhtesis code to study binaries hosting a Wolf-Rayet star and a black hole. I investigated their role as progenitors of merging binary black holes, comparing my results to the observed properties of Cyg X-3. 
    \end{itemize}
    }
  \emptySeparator
  \experience
    {Sep 2020}   {Bachelor in Astronomy}{University of Padua}
    {Oct 2017} 
    {Thesis: \textit{Impact of mass transfer efficiency on the formation of binary compact objects} \\
    Supervisor: Prof. Michela Mapelli; Co-Supervisor: Dr. Giuliano Iorio \\
    Grade: 110/110 cum laude  }{
    \begin{itemize}
        \item  By means of numerical simulations with the SEVN code, I studied the impact of mass transfer processes on the formation of binary compact objects, focusing on binaries merging via gravitational wave emission.
    \end{itemize}
   }
  \emptySeparator
    \experience
    {Jul 2017}   {Scientific High School ``G.B. Benedetti"}{Venice}
    {Sep 2012} 
    {Final project: \textit{The Pleiades} \\
    Grade: 100/100 cum laude }{    
    \begin{itemize}
        \item I calculated the distance of the Pleiades open cluster with the parallax method, reducing Hipparcos data with the TOPCAT software.
    \end{itemize}
    } 
  %\emptySeparator
\end{experiences}
