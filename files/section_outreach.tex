% YAAC Another Awesome CV LaTeX Template
%
% This template has been downloaded from:
% https://github.com/darwiin/yaac-another-awesome-cv
%
% Author:
% Christophe Roger
%
% Template license:
% CC BY-SA 4.0 (https://creativecommons.org/licenses/by-sa/4.0/)

%Section: Scholarships and additional info
\sectionTitle{Outreach}{\faChild}

\begin{experiences}
    \outreachlink{Now}{Oct 2023}{Museum guide, Padua} {}{
     \begin{itemize}
         \item After a training period (October 2023), I will guide the general public through the collection of scientific instruments hosted at the ``Giovanni Poleni'' Physics Museum of Padua.
     \end{itemize}}
    \emptySeparator
    \outreachlink{29 Sep 2023}{}{European Researcher's Night - Science4All, Padua} {\website{https://meetmetonight.oapd.inaf.it/programma}{Website}}{
     \begin{itemize}
         \item I explained my work as a researcher to the general public. I gave a talk (\emph{Olive Ascolane Stellari}) about my research topic and introduced families and kids to the science world.
     \end{itemize}}
    \emptySeparator
    \outreachlink{23,24 Mar 2023}{}{Science and mythology of constellations, Venice} {}{
     \begin{itemize}
         \item I gave two lessons at the scientific high school ``G.B. Benedetti'' in Venice to explain to the students the link between constellation's mythology and astronomical phenomena like Earth's rotation, revolution or axial precession. I used practical and digital tools, bringing spheres, gyroscopes and scripting a plugin in \emph{Stellarium} to automatize the sky visualization throughout centuries and nights.
     \end{itemize}}
    \emptySeparator
    \outreachlink{Jun 2023}{Nov 2022}{Science from the Islamic world to today's Europe, Padua} {\website{http://www.dfa.unipd.it/terza-missione/scienza-dal-mondo-islamico-alleuropa-di-oggi/dottorandi-corso-soft-skills/}{Website}}{
     \begin{itemize}
         \item I contributed to the creation of new outreach projects for the ``Giovanni Poleni'' Physics Museum of Padua, focusing on the communication of the scientific research and teaching practices brought from the Islamic world to today's Europe. The projects were developed by mixed working-groups, involving PhD students and members of Padua foreign communities.
     \end{itemize}}
    \emptySeparator
    \outreachlink{16 Jul 2019}{}{Telescope observations open to the general public, Padua} {\website{https://www.padovanet.it/evento/iniziativa-50-anni-dallo-sbarco-sulla-luna}{Website}}{
     \begin{itemize}
         \item I collaborated with the amateur astronomers of Padua, using their telescopes to illustrate celestial objects in the public event organized for the partial lunar eclipse.
     \end{itemize}}
     %\emptySeparator
\end{experiences}