% YAAC Another Awesome CV LaTeX Template
%
% This template has been downloaded from:
% https://github.com/darwiin/yaac-another-awesome-cv
%
% Author:
% Christophe Roger
%
% Template license:
% CC BY-SA 4.0 (https://creativecommons.org/licenses/by-sa/4.0/)

%Section: Scholarships and additional info
\sectionTitle{Outreach}{\faChild}

\begin{experiences}
    
    \outreachlink{Oct 2023 - Ongoing}{}{Museum guide at the ``Giovanni Poleni'' Physics Museum, Padua} {\website{https://www.musei.unipd.it/it/fisica}{Website}}{
     \begin{itemize}
         \item I accompany visitors through the exhibition of historical (18-19-20th century) scientific instruments hosted at the ``Giovanni Poleni'' Physics Museum in Padua.
     \end{itemize}}
    \emptySeparator
    \outreachlink{28, 29 Sep 2024}{29 Sep 2023}{European Researcher's Night - Science4All, Padua} {\website{https://meetmetonight.oapd.inaf.it/edizione-passate/edizione-2023}{Website}}{
     \begin{itemize}
         \item 2024: I collaborated to the activity ``Cartoline Stellari'', introducing kids to constellations.
         \item 2023: I gave a talk (\emph{Olive Ascolane Stellari}) about my research topic to the general public.
     \end{itemize}}
     \emptySeparator
    \outreachlink{25 Apr 2024}{}{Girl's Day, University of Heidelberg} {\website{https://www.physik.uni-heidelberg.de/girlsday}{Website}}{
     \begin{itemize}
         \item I prepared an a gravitational-wave-themed activity for the Girl's Day event. The initiative, held also at the University of Heidelberg, promoted STEM disciplines among young girls.
     \end{itemize}}
    \emptySeparator
    \outreachlink{Feb - Jun 2024}{}{Designing innovative public engagement activities, Castellaro Lagusello} {\website{https://www.astronomiacastellaro.oapd.inaf.it/}{Website}}{
     \begin{itemize}
         \item I developed the outreach project ``Caccia al ladro fra le stelle'' for the astronomy festival in Castellaro Lagusello (Italy). The project was developed to be accessible also to people with disabilities and is available at \link{}{https://play.inaf.it/caccia-al-ladro-fra-le-stelle/}
     \end{itemize}}
    \emptySeparator
    \outreachlink{Oct 2023 - Feb 2024}{}{Teacher for the outreach project ``Il cielo come laboratorio'', Venice} {\website{http://www.astro.unipd.it/progettoeducativo/programma.html}{Website}}{
     \begin{itemize}
         \item I held an astronomy course (16 h.) at the high school "Benedetti-Tommaseo" in Venice as part of the astronomy outreach project ``Il cielo come laboratorio'' coordinated by the University of Padua. The project offers university-orientation in scientific subjects to high school students. 4/30 of the students following my lessons were selected ($\sim 23 \%$ success rate) to participate in a 4-days stage at the Asiago observatory to analyze astronomical data.   %The course explains at a university-like level topics of quantum mechanics, stellar evolution, galaxy morphology, planetary and exoplanetary characterization, spectroscopic and photometric techniques. In addition, I also organized telescope observation nights with the students. Eventually, 4/30 of the students following my lessons were selected to participate in a 4-days stage at the Asiago observatory (regional selection for 24/255 candidates) where they team-worked to analyze astronomical data, in a researcher-like experience.
     \end{itemize}}
    \emptySeparator
    \outreachlink{10 Nov 2023}{}{``Geppina Coppola'' prize, INAF-OACN Naples} {\website{https://www.associazionegeppinacoppola.it/premio-gc/}{Website}}{
     \begin{itemize}
         \item I presented my master thesis work to the general public in the ceremony where I was awarded the ``Geppina Coppola'' prize for best master thesis in astronomy in Italy.
     \end{itemize}}
     \emptySeparator
    \outreachlink{23, 24 Mar 2023}{}{Science and mythology of constellations, Venice} {}{
     \begin{itemize}
         \item I held two lessons to students in the high school ``G.B. Benedetti'' in Venice, explaining the link between constellation's mythology and astronomical phenomena. %like Earth's rotation, revolution or axial precession. 
         I brought spheres and gyroscopes in class, and scripted a plugin in \emph{Stellarium} to automatize the sky visualization.
     \end{itemize}}
    \emptySeparator
    \outreachlink{Nov 2022 - Jun 2023}{}{Science from the Islamic world to today's Europe, Padua} {\website{http://www.dfa.unipd.it/terza-missione/scienza-dal-mondo-islamico-alleuropa-di-oggi/dottorandi-corso-soft-skills/}{Website}}{
     \begin{itemize}
         \item I worked with PhD students and members of the Padua foreign communities to create new outreach projects for the ``Giovanni Poleni'' Physics Museum of Padua, communicating the scientific research and teaching practices brought from the Islamic world to today's Europe.
     \end{itemize}}
    \emptySeparator
    \outreachlink{16 Jul 2019}{}{Telescope observations open to the general public, Padua} {\website{https://www.padovanet.it/evento/iniziativa-50-anni-dallo-sbarco-sulla-luna}{Website}}{
     \begin{itemize}
         \item I collaborated with the amateur astronomers of Padua, using their telescopes to illustrate celestial objects in the public event organized for the partial lunar eclipse.
     \end{itemize}}
     %\emptySeparator
\end{experiences}