% YAAC Another Awesome CV LaTeX Template
%
% This template has been downloaded from:
% https://github.com/darwiin/yaac-another-awesome-cv
%
% Author:
% Christophe Roger
%
% Template license:
% CC BY-SA 4.0 (https://creativecommons.org/licenses/by-sa/4.0/)

%Section: Scholarships and additional info
\sectionTitle{Outreach}{\faChild}

\begin{experiences}
    
    \outreachlink{Ongoing}{Oct 2023}{Museum guide, Padua} {\website{https://www.musei.unipd.it/it/fisica}{Website}}{
     \begin{itemize}
         \item I accompany students, families and visitors through the exhibition of scientific instruments hosted at the ``Giovanni Poleni'' Physics Museum in Padua.
     \end{itemize}}
    \emptySeparator
    \outreachlink{Ongoing}{Feb 2024}{Designing innovative public engagement activities, Padua} {\website{https://www.astronomiacastellaro.oapd.inaf.it/}{Website}}{
     \begin{itemize}
         \item Among with other PhD students in astronomy of the University of Padua, I am developing outreach projects for the astronomy festival that will be held in June in Castellaro Lagusello (Italy). The projects are aimed to kids and are accessible also to people with disabilities.
     \end{itemize}}
    \emptySeparator
    \outreachlink{Feb 2024}{Oct 2023}{Teacher for the outreach project ``Il cielo come laboratorio'', Venice} {\website{http://www.astro.unipd.it/progettoeducativo/programma.html}{Website}}{
     \begin{itemize}
         \item I held an astronomy course at the high school "Benedetti-Tommaseo" in Venice as part of the astronomy outreach project ``Il cielo come laboratorio'' coordinated by the University of Padua. The project aims to offer university-orientation in scientific subjects to high school students. The course explains at a university-like level topics of quantum mechanics, stellar evolution, galaxy morphology, planetary and exoplanetary characterization, spectroscopic and photometric techniques. In addition, I also organized telescope observation nights with the students. Eventually, 4/30 of the students following my lessons were selected to participate in a 4-days stage at the Asiago observatory (regional selection for 24/255 candidates) where they team-worked to analyze astronomical data, in a researcher-like experience.
     \end{itemize}}
    \emptySeparator
    \outreachlink{10 Nov 2023}{}{``Geppina Coppola'' prize, INAF-OACN Naples} {\website{https://www.associazionegeppinacoppola.it/premio-gc/}{Website}}{
     \begin{itemize}
         \item I explained my master thesis work to the general public in the ceremony where I was awarded the ``Geppina Coppola'' prize for best master thesis in astronomy in Italy.
     \end{itemize}}
     \emptySeparator
    \outreachlink{29 Sep 2023}{}{European Researcher's Night - Science4All, Padua} {\website{https://meetmetonight.oapd.inaf.it/programma}{Website}}{
     \begin{itemize}
         \item I explained my work as a researcher to the general public. I gave a talk (\emph{Olive Ascolane Stellari}) about my research topic and introduced families and kids to the science world.
     \end{itemize}}
    \emptySeparator
    \outreachlink{23,24 Mar 2023}{}{Science and mythology of constellations, Venice} {}{
     \begin{itemize}
         \item I gave two lessons at the scientific high school ``G.B. Benedetti'' in Venice to explain to the students the link between constellation's mythology and astronomical phenomena like Earth's rotation, revolution or axial precession. I used practical and digital tools, bringing spheres, gyroscopes and scripting a plugin in \emph{Stellarium} to automatize the sky visualization throughout centuries and nights.
     \end{itemize}}
    \emptySeparator
    \outreachlink{Jun 2023}{Nov 2022}{Science from the Islamic world to today's Europe, Padua} {\website{http://www.dfa.unipd.it/terza-missione/scienza-dal-mondo-islamico-alleuropa-di-oggi/dottorandi-corso-soft-skills/}{Website}}{
     \begin{itemize}
         \item I contributed to the creation of new outreach projects for the ``Giovanni Poleni'' Physics Museum of Padua, focusing on the communication of the scientific research and teaching practices brought from the Islamic world to today's Europe. The projects were developed by mixed working-groups, involving PhD students and members of Padua foreign communities.
     \end{itemize}}
    \emptySeparator
    \outreachlink{16 Jul 2019}{}{Telescope observations open to the general public, Padua} {\website{https://www.padovanet.it/evento/iniziativa-50-anni-dallo-sbarco-sulla-luna}{Website}}{
     \begin{itemize}
         \item I collaborated with the amateur astronomers of Padua, using their telescopes to illustrate celestial objects in the public event organized for the partial lunar eclipse.
     \end{itemize}}
     %\emptySeparator
\end{experiences}