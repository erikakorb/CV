% YAAC Another Awesome CV LaTeX Template
%
% This template has been downloaded from:
% https://github.com/darwiin/yaac-another-awesome-cv
%
% Author:
% Christophe Roger
%
% Template license:
% CC BY-SA 4.0 (https://creativecommons.org/licenses/by-sa/4.0/)

%Section: Scholarships and additional info
\sectionTitle{Outreach}{\faChild}

\begin{experiences}
    \outreachentry{Physics Museum}{``Giovanni Poleni''}{Museum guide}{Padua} {Oct 2023 - Ongoing}%\website{https://www.musei.unipd.it/it/fisica}{Website}}{
    {
     \begin{itemize}
         \item I accompany visitors through the exhibition of historical (18-19-20th century) scientific instruments hosted at the ``Giovanni Poleni'' Physics Museum in Padua.
     \end{itemize}}
     \\
    \outreachentrynocity{ }{ }{Project ``Science from the Islamic world to today's Europe''}{Nov 2022 - Jun 2023} %{\website{http://www.dfa.unipd.it/terza-missione/scienza-dal-mondo-islamico-alleuropa-di-oggi/dottorandi-corso-soft-skills/}{Website}}
    {
     \begin{itemize}
         \item I collaborated with PhD students and members of the Padua foreign communities to promote the instrument collection of the ``Giovanni Poleni'' Physics Museum of Padua. We designed new outreach projects to communicate scientific researches and teaching practices brought from the Islamic world to today's Europe.
     \end{itemize}}

    \emptySeparator
    \outreachentry{Public Engagement}{Activities}{Activity designer and facilitator for astronomical festivals}{Castellaro Lagusello}{} %\website{https://www.astronomiacastellaro.oapd.inaf.it/}{Website}}
    {
     \begin{itemize}
         \item I designed and presented multisensorial activities for the astronomy festival in Castellaro Lagusello (Italy) (\link{}{https://www.astronomiacastellaro.oapd.inaf.it/}). The activities were developed to be accessible also to people with visual and hearing impairments.
         \item Light pollution activity (ongoing) \hfill Jan - Jun 2025
         \item Astrolabe activity: \link{}{https://play.inaf.it/caccia-al-ladro-fra-le-stelle/} \hfill Feb - Jun 2024
     \end{itemize}}
    \\
    \outreachentry{}{}{Facilitator at European Researcher's Night - Science4All}{Padua} {28, 29 Sep 2024}%\website{\website{https://meetmetonight.oapd.inaf.it/edizione-passate/edizione-2023}{Website}}{
    {
     \begin{itemize}
         \item Facilitator in the hand-on activity ``Cartoline Stellari'', to introduce kids to constellations.
     \end{itemize}}
     \\
    \outreachentry{ }{ }{Activity designer and facilitator for the ``Girl's Day''}{Heidelberg}{25 Apr 2024} %{\website{https://www.physik.uni-heidelberg.de/girlsday}{Website}}
    {
     \begin{itemize}
         \item I designed a gravitational-wave-themed activity for the Girl's Day event. The initiative, held at the University of Heidelberg, promoted STEM disciplines among young girls.
     \end{itemize}}
     \\
    \outreachentry{}{}{Facilitator with amateur astronomers}{Padua}{16 Jul 2019} %{\website{https://www.padovanet.it/evento/iniziativa-50-anni-dallo-sbarco-sulla-luna}{Website}}
    {
     \begin{itemize}
         \item I collaborated with the amateur astronomers of Padua, using their telescopes to illustrate celestial objects in a public event organized for a partial lunar eclipse.
     \end{itemize}}
    \emptySeparator
         
    \outreachentry{High School}{Lecture Courses}{Main teacher of the course ``Il cielo come laboratorio''}{ Venice}{Oct 2023 - Feb 2024} %{\website{http://www.astro.unipd.it/progettoeducativo/programma.html}{Website}}
    {
     \begin{itemize}
         \item I held an astronomy course (16 h.) at the high school "Benedetti-Tommaseo" in Venice as part of the astronomy outreach project ``Il cielo come laboratorio'' coordinated by the University of Padua. The project offers university-orientation in scientific subjects to high school students. 4/30 of the students following my lessons were selected ($\sim 23 \%$ success rate) to participate in a 4-days stage at the Asiago observatory to analyze astronomical data.   %The course explains at a university-like level topics of quantum mechanics, stellar evolution, galaxy morphology, planetary and exoplanetary characterization, spectroscopic and photometric techniques. In addition, I also organized telescope observation nights with the students. Eventually, 4/30 of the students following my lessons were selected to participate in a 4-days stage at the Asiago observatory (regional selection for 24/255 candidates) where they team-worked to analyze astronomical data, in a researcher-like experience.
     \end{itemize}}
     \\
    \outreachentry{}{}{Teacher on ``Science and mythology of constellations''}{Venice}{23, 24 Mar 2023}{
     \begin{itemize}
         \item I designed and held two lessons (4 h.) to students in the high school ``Benedetti-Tommaseo'' in Venice. I explained the link between constellation's mythology and astronomical phenomena, bringing spheres and gyroscopes in class, and scripting an ad-hoc plugin in \emph{Stellarium}.
     \end{itemize}}
    \emptySeparator

    \newpage
    \outreachentrynocity{Public Talks}{}{Research activity presentations to the general public}{}
    {
     \begin{itemize}
         \item To the amateur astronomers of Treviso (Italy) \hfill 10 Jan 2025
         \item For the ``Geppina Coppola'' prize at INAF-OACN, Naples  \hfill 10 Nov 2023 %\website{https://www.associazionegeppinacoppola.it/premio-gc/
         \item For the European Researcher's Night - Science4All in Padua  \hfill 23 Sep 2023
     \end{itemize}}
\end{experiences}